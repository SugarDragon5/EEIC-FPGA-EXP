\subsection{スケジュールと実装内容}
まず、実験のスケジュールと実装内容を表\ref{tab:schedule}に示す。
\begin{table}
    \begin{center}
        \caption{実験のスケジュールと実装内容}\label{tab:schedule}
        \begin{tabular}{|c|c|c|} \hline
            日付 & 実装内容 & 備考 \\ \hline
            10/12 & Pythonで山登り法・焼き鈍し法プログラムを実装 &  \\ \hline
            10/16 & 乱数計算・距離計算モジュールを実装 &  \\ \hline
            10/17 & 近傍生成・山登り法モジュールを実装 &  \\ \hline
            10/19 & ロジックのバグ修正・テストベンチ上で山登り法完動 &  \\ \hline
            10/23 & 山登り法モジュールの並列化 &  \\ \hline
            10/24, 26, 27 & FPGA上でうまく動作しないバグの修正 &  \\ \hline
        \end{tabular}
    \end{center}
\end{table}
\ref{sec:goal}節で述べた目標のうち、1. 2. について実行し、FPGA上で山登り法を並列に実行することで高速に巡回セールスマン問題の局所探索法を求めることができた。
次項から、実装したハードウエアの構成・アルゴリズムを説明する。

\subsection{ハードウエア構成}
\begin{figure}
    \caption{ハードウエア構成}\label{fig:hardware}
    \begin{center}
        \includegraphics[width=15cm]{figure/hardware.png}
    \end{center}
\end{figure}

図~\ref{fig:hardware}にハードウエアの構成を示す。
FPGAには以下のモジュールを実装した。
\begin{itemize}
    \item メインモジュール(山登り法)
    \item グラフ生成モジュール(近傍生成)
    \item 交換チェックモジュール(交換判定)
    \item 乱数生成モジュール
    \item 距離計算モジュール
\end{itemize}
実行の流れは、
\begin{enumerate}
    \item リセット入力がなされると、グラフ生成モジュールが乱数生成モジュールを用いてランダムな頂点・経路を生成する
    \item メインモジュールが近傍を生成する。
    \item 交換チェックモジュールが距離計算モジュールを用いて交換の有無を判定する。
    \item メインモジュールが交換チェックモジュールからの結果を受け取り、交換の有無に応じて経路を更新する。
    \item 2. に戻る。
\end{enumerate}
である。2. ~ 4. は並列に実行される。

\subsection{乱数生成アルゴリズム}
今回の問題では、ランダムなグラフの生成・近傍の生成など、乱数を用いる箇所が多い。
FPGA上で疑似乱数を生成するためのアルゴリズムとして、今回はXORShiftを用いた。
XORShiftは状態$x, y, z, w$に対し、
\begin{align*}
    x \leftarrow& y \\
    y \leftarrow& z \\
    z \leftarrow& w \\
    w \leftarrow& (w\oplus(w>>19))\oplus((x\oplus(x<<11))\oplus((x\oplus(x<<11))>>8))
\end{align*}
を計算することで疑似乱数列${w}$を求めるアルゴリズムである。
ビット演算のみで計算できるため極めて高速で、また周期も$2^{128}-1$と非常に長く品質も高いという特徴がある。
このアルゴリズムを用いて、クロック毎に新しい乱数を生成するモジュールを作成し、乱数が必要な箇所それぞれでインスタンス化した。

\subsection{距離計算アルゴリズム}\label{sec:distance}
距離計算モジュールは、2つの頂点のx, y座標$(x_1, y_1), (x_2, y_2)$を受け取り、
2点のユークリッド距離$\sqrt{(x_1-x_2)^2+(y_1-y_2)^2}$を計算するモジュールである。

このモジュールでは、加算減算乗算に加え、平方根の計算が必要となるが、今回はニュートン法を用いて計算した。
ニュートン法において、$\sqrt{a}$の計算のためには、$f(x)=x^2-a$を考える。このとき、$f(x)=0$となる$x$が$\sqrt{a}$である。

解に近い値を初期値$x_0$とし、$x_{n+1}=x_n-\frac{f(x_n)}{f'(x_n)}=x_n-\frac{x_n^2-a}{2x_n}=\frac{x_n}{2}+\frac{a}{2x_n}$に従って$x_n$を更新していくと、
$x_n$は$\sqrt{a}$に収束していく。
今回はユークリッド距離の計算であり、初期値としてマンハッタン距離$d_m=|x_1-x_2|+|y_1-y_2|$を用いると、比較的早く収束する。
x座標、y座標が$[0,255]$の2点に対し、整数範囲で誤差1以内の距離を計算するには、$x_3$まで計算すれば十分であることがCPUプログラムによる検証により判明した。

当初、1クロックで$x_3$までの計算(二乗距離・マンハッタン距離・ニュートン法3回)を実行していたが、後述の通り周波数が大きく低下してしまったため、図~\ref{fig:distance}のように5クロックに分割しパイプライン化した。

\begin{figure}
    \begin{center}
        \includegraphics[width=15cm]{figure/distance_newton.jpg}
        \caption{距離計算モジュールのパイプライン化}\label{fig:distance}
    \end{center}
\end{figure}

\subsection{近傍生成アルゴリズム}
\ref{sec:local}節で述べたように、今回は経路の2点を入れ替えることで近傍を生成する。
交換する2点を選ぶには、交換する2点と、それぞれの前後の点の計6点の情報が必要であり、これら6点が重複しないように選択すれば、複数の箇所について同時に近傍を生成し、判定・更新を行うことができる。

各モジュールが頂点を選択する際は、以下の手順で行う
\begin{enumerate}
    \item 乱数生成モジュールから、頂点番号を表す[0,63]の乱数を2つ受け取り、$v_1, v_2$とする。
    \item ロック配列(後述)の$v_1-1, v_1, v_1+1, v_2-1, v_2, v_2+1$番目の要素を確認し、いずれも他のモジュールによってロックされていなければ、ロックを試みる。
    \item 再び$v_1-1, v_1, v_1+1, v_2-1, v_2, v_2+1$番目の要素を確認し、いずれもロックできていれば、交換チェックモジュールに6頂点の座標を送信する。
          一つ以上の頂点がロックに失敗していれば、6頂点すべてのロックを解除し、1. に戻る。
\end{enumerate}
ロック配列は、各頂点が他のモジュールによってロックされているかどうかを表す配列である。(ロックされていなければ-1, ロックされていれば、ロックしているモジュールのIDが格納される。)
この機能によって、複数のモジュールが同時に同じ頂点を選択することを防ぐことができる。
この手法は排他制御ロックと呼ばれており、データベースのトランザクション処理や、マルチスレッドプログラミングにおいてよく用いられる。

\subsection{交換判定アルゴリズム}
交換判定モジュールは、選択した2頂点(p,q)とその前後の頂点の座標を受け取り、距離計算モジュールを用いて交換するべきかどうか(交換によって総距離が短くなるかどうか)を判定する。

\fbox{パターン1}\hspace{10pt}p,qが隣接していない場合、p, qと前後を含めた6点がa→p→bとc→q→dの順に並んでいるとして、pとqを交換するので、
\begin{itemize}
    \item 交換前: $d(a,p)+d(p,c)+d(d,q)+d(q,f)$
    \item 交換後: $d(a,q)+d(q,c)+d(d,p)+d(p,f)$
\end{itemize}
の2つの距離を計算し、交換後の方が短ければ交換する。

\fbox{パターン2}\hspace{10pt}p,qが隣接している場合、p, qと前後を含めた4点がa→p→q→bの順に並んでいるとして、pとqを交換するので、
\begin{itemize}
    \item 交換前: $d(a,p)+d(p,q)+d(q,f)$
    \item 交換後: $d(a,q)+d(q,p)+d(p,f)$
\end{itemize}
を比較する。$d(p,q)=d(q,p)$より$d(a,p)+d(q,f)$と$d(a,q)+d(p,f)$を計算し、後者の方が小さければ交換する。

山登りモジュールではパターン1とパターン2をそれぞれいくつか用意し、並列度を上げた。

