\subsection{合成結果全体に対する検討}


\subsection{最大動作周波数に関する分析・追加検証}
\ref{sec:distance}節で述べたように、距離計算モジュールは5クロックに分割しパイプライン化した。
当初1クロックで計算していた際、最大周波数が4MHz程度しか出なかったが、5クロックに分割したことで12MHz程度に改善した。
また、ユークリッド距離のかわりにマンハッタン距離を用いた場合、47.8MHzで動作可能であることを確認した。

ステージ分割によって、1クロックあたりの除算の回数が3回から最大1回に減少し、最大動作周波数は三倍程度になった。
このことから、最大動作周波数のボトルネックは距離計算モジュールであり、特に除算に大きな問題があることがわかった。

除算は、加算減算乗算に比べ非常に遅く、一般的なプロセッサでは、数十クロックかけることもある。したがって、除算を含むモジュールは、動作周波数を大きく下げる要因となる。
この問題を是正するために、以下の2つの方法を考え、それぞれ検証した。
\begin{itemize}
    \item 方法1: 除算の高速化
    \item 方法2: 除算を含まない距離計算方法の実装
\end{itemize}
\subsubsection{方法1: 除算の高速化}
今回除算を必要とするのは、二乗距離の平方根を計算する部分であり、各座標が$[0,255]$の8bit整数であることから、二乗距離は18bitで表現可能である。
現状32bitに対する除算を用いていたことから、除算の精度を下げることで高速化を図ることができると考えた。

除算の精度を18bitに下げたところ、最大動作周波数は21.4MHzまで改善した。

\subsubsection{方法2: 除算を含まない距離計算方法の実装}
今回の平方根計算ではニュートン法を用いたが、平方根計算には除算が不要な方法も存在する。
その中でも、二分法を用いて距離計算する方法を検討する。
ある値$t$と$\sqrt{x}$の大小関係は、$t^2$と$x$の大小関係と同じであることを利用し、上位ビットから順に1ビットずつ決定することができる。

例えば、$x=100 (\sqrt{100}=10)$に対して、
\begin{itemize}
    \item $t=10000_{(2)}=16$のとき、$t^2=100000000_{(2)}=256$より、$t^2>x$であるため、$t$の5ビット目は0となる。
    \item $t=01000_{(2)}=8$のとき、$t^2=010000000_{(2)}=64$より、$t^2<x$であるため、$t$の4ビット目は1となる。
    \item $t=01100_{(2)}=12$のとき、$t^2=011000000_{(2)}=144$より、$t^2>x$であるため、$t$の3ビット目は0となる。
    \item $t=01010_{(2)}=10$のとき、$t^2=010100100_{(2)}=100$より、$t^2=x$であるため、$t$の2ビット目は1となる。
    \item $t=01011_{(2)}=11$のとき、$t^2=010110100_{(2)}=121$より、$t^2>x$であるため、$t$の1ビット目は0となる。
\end{itemize}
以上より、$\sqrt{100}=01010_{(2)}=10$が求まる。二分法では、$O(\log x)$回の乗算と比較で平方根を求めることができる。

今回、ユークリッド距離は9bitで表現可能なので、9回の乗算と比較で整数範囲で正確な距離を求めることができる。

この方法を用いて、11クロックでパイプライン化して計算するよう変更を加えると、最大動作周波数は48.5MHzまで改善した。



