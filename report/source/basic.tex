\subsection{基本課題1}
\subsubsection{実験内容}
\begin{itemize}
    \item 課題1-1: 配布資料にあるプログラムEXE1をQuartusでコンパイルし、FPGA上で実行した。
    \item 課題1-2: EXE1で実装されたルーレットプログラムを改変し、点灯パタンを変更した。
    \begin{enumerate}
        \item 7セグの複数のセグメントを点灯させた。
        \item ルーレットの回転方向が一定周期で時計周り・反時計回りを繰り返すようにした。
        \item 7セグの中央のセグメントを一定のリズムでチカチカさせた。
    \end{enumerate}
\end{itemize}
\subsubsection{実験結果}
課題1-1, 1-2のそれぞれに対し、Quartusでのコンパイル時に回路規模(必要ロジック数)と最大動作クロック周波数を測定した結果を表~\ref{tab:basic1}に示す。

\begin{table}[h]
    \caption{課題1-1(左), 1-2(右)における回路規模と最大動作クロック周波数}\label{tab:basic1}
    \begin{tabular}{|l|l|}
    \hline
    Estimate of Logic utilization (ALMs needed) & 24 \\\hline
    Combinational ALUT usage for logic          & 43 \\\hline
    -- 7 input functions                        & 0  \\\hline
    -- 6 input functions                        & 4  \\\hline
    -- 5 input functions                        & 0  \\\hline
    -- 4 input functions                        & 1  \\\hline
    -- \textless{}=3 input functions            & 38 \\\hline
    Dedicated logic registers                   & 39 \\\hline
    I/O pins                                    & 15 \\\hline
    Total DSP Blocks                            & 0  \\\hline
    Fmax [MHz]                                       & 389.71 \\\hline
    \end{tabular}
    \begin{tabular}{|l|l|}
        \hline
        Estimate of Logic utilization (ALMs needed) & 24 \\\hline
        Combinational ALUT usage for logic          & 42 \\\hline
        -- 7 input functions                        & 0  \\\hline
        -- 6 input functions                        & 5  \\\hline
        -- 5 input functions                        & 5  \\\hline
        -- 4 input functions                        & 0  \\\hline
        -- \textless{}=3 input functions            & 32 \\\hline
        Dedicated logic registers                   & 39 \\\hline
        I/O pins                                    & 15 \\\hline
        Total DSP Blocks                            & 0 \\\hline
        Fmax [MHz]                                       & 323.21 \\\hline
    \end{tabular}
\end{table}

課題1-1 → 1-2で処理が少し複雑になったことによって、
\begin{itemize}
    \item 回路規模は大きく変化しなかった。
    \item 必要な多入力演算器の数が多くなった。
    \item 最大動作クロック周波数が低下した。
\end{itemize}

\subsection{基本課題2}
\subsubsection{実験内容}
配布資料にあるプログラムEXE2(電卓・リモコン制御プログラム)をQuartusでコンパイルし、FPGA上で実行した。
\subsubsection{実験結果}
EXE2をコンパイルした際の回路規模(必要ロジック数)と最大動作クロック周波数を表~\ref{tab:basic2}に示す。
\begin{table}[h]
    \caption{課題2における回路規模と最大動作クロック周波数}\label{tab:basic2}
    \begin{center}
    \begin{tabular}{|l|l|}
    \hline
    Estimate of Logic utilization (ALMs needed) & 576  \\\hline
    Combinational ALUT usage for logic          & 1109 \\\hline
    -- 7 input functions                        & 6    \\\hline
    -- 6 input functions                        & 32   \\\hline
    -- 5 input functions                        & 154  \\\hline
    -- 4 input functions                        & 200  \\\hline
    -- \textless{}=3 input functions            & 717  \\\hline
    Dedicated logic registers                   & 188  \\\hline
    I/O pins                                    & 69   \\\hline
    Total DSP Blocks                            & 1   \\\hline
    Fmax [MHz]                                  & 33.54 \\\hline
    \end{tabular}
    \end{center}
\end{table}
課題1に比べると、ロジックの規模が非常に大きくなり、DSPも使用するようになった。
また、クロック周波数が大幅に低下している。クリティカルパスは確認していないが、おそらく四則演算(特に除算)がボトルネックになっていると思われる。
\subsection{基本課題3}
\subsubsection{実験内容}
\begin{itemize}
    \item 課題3-1: 配布資料にあるプログラムEXE31(VGA出力プログラム)をQuartusでコンパイルし、FPGA上で実行した。
    \item 課題3-2: 3-1のプログラムから、出力されるパタンを細分化(8x8 →16x16)した。
    \item 課題3-3: 3-2のプログラムに加え、左上に自分の名前(Ryugo.S)と表示させた。
\end{itemize}
\subsubsection{実験結果}
課題3-1のプログラムをコンパイルした際の回路規模(必要ロジック数)と最大動作クロック周波数を表~\ref{tab:basic3}に示す。
3-2, 3-3については、3-1と大きな差が生じなかったので省略する。
\begin{table}[h]
    \caption{課題3-1における回路規模と最大動作クロック周波数}\label{tab:basic3}
    \begin{center}
    \begin{tabular}{|l|l|}
        \hline
    Estimate of Logic utilization (ALMs needed) & 118    \\\hline
    Combinational ALUT usage for logic          & 190    \\\hline
    -- 7 input functions                        & 0      \\\hline
    -- 6 input functions                        & 45     \\\hline
    -- 5 input functions                        & 10     \\\hline
    -- 4 input functions                        & 13     \\\hline
    -- \textless{}=3 input functions            & 122    \\\hline
    Dedicated logic registers                   & 95     \\\hline
    I/O pins                                    & 96     \\\hline
    Total MLAB memory bits                      & 0      \\\hline
    Total block memory bits                     & 589824 \\\hline
    Total DSP Blocks                            & 0      \\\hline
    Total PLLs                                  & 1      \\\hline
    -- PLLs                                     & 1     \\\hline
    PLL[MHz] & 101.08\\\hline
    VGA[MHz] & 192.46\\\hline
    \end{tabular}
    \end{center}
\end{table}

画面出力にあたり、各ピクセルごとにRGBの値を保持しておく必要があり、これをblockメモリで実装しているため利用量が大きくなっている。
VGA出力は40MHzのクロックで動作するため、PLLを挟む必要があり、2つのクロックが同時に動作するような格好になる。それぞれに対し最大動作周波数を考えることになる。

\subsection{基本課題4}
\subsubsection{実験内容}
\begin{itemize}
    \item 課題4-1: 配布資料にあるプログラムEXE41(音声入出力プログラム)をQuartusでコンパイルし、FPGA上で実行した。
    \item 課題4-2: 配布資料にあるプログラムEXE42(周波数の倍化・半化プログラム)をQuartusでコンパイルし、FPGA上で実行した。
\end{itemize}
\subsubsection{実験結果}
課題4-1のプログラムをコンパイルした際の回路規模(必要ロジック数)と最大動作クロック周波数を表~\ref{tab:basic4}に示す。
\begin{table}[h]
    \caption{課題4-1における回路規模と最大動作クロック周波数}\label{tab:basic4}
    \begin{center}
    \begin{tabular}{|l|l|}
    \hline
    Estimate of Logic utilization (ALMs needed) & 162    \\\hline
    Combinational ALUT usage for logic          & 262    \\\hline
    -- 7 input functions                        & 6      \\\hline
    -- 6 input functions                        & 31     \\\hline
    -- 5 input functions                        & 20     \\\hline
    -- 4 input functions                        & 33     \\\hline
    -- \textless{}=3 input functions            & 172    \\\hline
    Dedicated logic registers                   & 178    \\\hline
    I/O pins                                    & 106    \\\hline
    Total MLAB memory bits                      & 0      \\\hline
    Total block memory bits                     & 196608 \\\hline
    Total DSP Blocks                            & 0      \\\hline
    CLOCK\_500:u1|COUNTER\_500[9]	    [MHz] & 160.82 \\\hline
    AUD\_BCLK	    [MHz] & 186.15 \\\hline
    CLOCK\_50	    [MHz] & 287.19 \\\hline
    i2c:u2|END  [MHz] & 418.76 \\\hline
    \end{tabular}
    \end{center}
\end{table}