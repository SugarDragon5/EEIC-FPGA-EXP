\subsection{基本課題1}
\subsubsection{実験内容}
\begin{itemize}
    \item 課題1-1: 配布資料にあるプログラムEXE1をQuartusでコンパイルし、FPGA上で実行した。
    \item 課題1-2: EXE1で実装されたルーレットプログラムを改変し、点灯パタンを変更した。
    \begin{enumerate}
        \item 7セグの複数のセグメントを点灯させた。
        \item ルーレットの回転方向が一定周期で時計周り・反時計回りを繰り返すようにした。
        \item 7セグの中央のセグメントを一定のリズムでチカチカさせた。
    \end{enumerate}
\end{itemize}
\subsubsection{実験結果}
課題1-1, 1-2のそれぞれに対し、Quartusでのコンパイル時に回路規模(必要ロジック数)と最大動作クロック周波数を測定した。結果は以下の通りである。
//TODO

課題1-1 → 1-2で処理が少し複雑になったことによって、
\begin{itemize}
    \item 回路規模は大きく変化しなかった。
    \item 必要な多入力演算器の数が多くなった。
\end{itemize}

\subsection{基本課題2}
\subsubsection{実験内容}
配布資料にあるプログラムEXE2(電卓・リモコン制御プログラム)をQuartusでコンパイルし、FPGA上で実行した。
\subsubsection{実験結果}
//TODO
\subsection{基本課題3}
\subsubsection{実験内容}
\begin{itemize}
    \item 課題3-1: 配布資料にあるプログラムEXE31(VGA出力プログラム)をQuartusでコンパイルし、FPGA上で実行した。
    \item 課題3-2: 3-1のプログラムから、出力されるパタンを細分化(8x8 →16x16)した。
    \item 課題3-3: 3-2のプログラムに加え、左上に自分の名前(Ryugo.S)と表示させた。
\end{itemize}
\subsubsection{実験結果}
//TODO

\subsection{基本課題4}
\subsubsection{実験内容}
\begin{itemize}
    \item 課題4-1: 配布資料にあるプログラムEXE41(音声入出力プログラム)をQuartusでコンパイルし、FPGA上で実行した。
    \item 課題4-2: 配布資料にあるプログラムEXE42(周波数の倍化・半化プログラム)をQuartusでコンパイルし、FPGA上で実行した。
\end{itemize}
\subsubsection{実験結果}
//TODO