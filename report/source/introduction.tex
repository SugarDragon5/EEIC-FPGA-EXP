本実験では、ハードウエア記述言語を用いてアルゴリズムを実装し、FPGA上で実行させることを目標とした。
現在、汎用プロセッサの性能は伸び悩みつつある。クロック周波数や集積度を上げることで性能を向上させることは難しくなってきていて、
近年はGPU (Graphic Processing Unit) など特定の処理に特化したデバイスを用いるアプローチが大きな成果を上げている。
FPGAはそのようなデバイスの一つであり、ハードウエア記述言語を用いて回路を記述・合成することで、特定の問題に特化したデバイスを作成することができる。

FPGAのようなハードウエア実装では、ソフトウエア実装にくらべ並列度・クロック周波数などを比較的容易に調整することができる。
よって、適切な実装を行えば、CPUより高速に問題を解決できるポテンシャルがある。
今回の実験では、基本課題としてVerilog HDL・FPGAの動作を学んだ後に
自由課題として一つの問題に対してソフトウエア実装とハードウエア実装を行い、性能比較を行った。
その結果をもとに、ハードウエアの特性やソフトとの違いを理解し、ハードウエア実装のポテンシャルについて考察した。

